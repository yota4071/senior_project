\chapter*{付録}
\addcontentsline{toc}{chapter}{付録}

\section*{付録1: ATMノード(Raspberry Pi)のプログラム}
\addcontentsline{toc}{section}{付録1: ATMノード(Raspberry Pi)のプログラム}

\begin{lstlisting}[language=Python, caption=ATMノードのプログラム]
# ここにATMノード(Raspberry Pi)のプログラムを記述してください

\end{lstlisting}

\section*{付録2: 店舗ノード(Jetson)のプログラム}
\addcontentsline{toc}{section}{付録2: 店舗ノード(Jetson)のプログラム}

\subsection*{2.1 開発環境}

\subsubsection*{ハードウェア}
\begin{itemize}
  \item NVIDIA Jetson Orin Nano(Linux 5.15.148-tegra)
  \item USBカメラ(640x480解像度)
\end{itemize}

\subsubsection*{ソフトウェア・ライブラリ}
\begin{itemize}
  \item Python 3.10
  \item Node.js v18.x
  \item Circom 2.0.0(ZK回路コンパイラ)
  \item snarkjs(Groth16証明生成・検証)
  \item Hardhat 2.22.3(Ethereumスマートコントラクト開発)
  \item ethers.js 6.16.0(ブロックチェーン通信)
  \item OpenCV(画像処理)
  \item PyTorch + CUDA(GPU推論)
  \item Ultralytics YOLOv8(物体検出)
  \item DeepSORT(マルチオブジェクトトラッキング)
  \item Flask(HTTPサーバ)
\end{itemize}

\subsubsection*{ブロックチェーン}
\begin{itemize}
  \item Ethereum Sepolia テストネット
  \item Solidity 0.8.28
\end{itemize}

\subsubsection*{Node.js依存パッケージ(project\_root/package.json)}
\begin{lstlisting}[language=json, caption=project\_root/package.json]
{
  "dependencies": {
    "circomlib": "^2.0.5"
  },
  "name": "project_root",
  "version": "1.0.0"
}
\end{lstlisting}

\subsubsection*{Node.js依存パッケージ(blockchain\_root/package.json)}
\begin{lstlisting}[language=json, caption=blockchain\_root/package.json]
{
  "name": "hardhat-project",
  "devDependencies": {
    "@nomicfoundation/hardhat-chai-matchers": "^2.1.0",
    "@nomicfoundation/hardhat-ethers": "^3.1.3",
    "@nomicfoundation/hardhat-network-helpers": "^1.1.2",
    "@nomicfoundation/hardhat-toolbox": "^3.0.0",
    "@nomicfoundation/hardhat-verify": "^1.1.1",
    "@typechain/ethers-v6": "^0.4.3",
    "@typechain/hardhat": "^8.0.3",
    "chai": "^4.5.0",
    "ethers": "^6.16.0",
    "hardhat": "^2.22.3",
    "hardhat-gas-reporter": "^1.0.10",
    "solidity-coverage": "^0.8.17",
    "ts-node": "^10.9.2",
    "typechain": "^8.3.2",
    "typescript": "^5.9.3"
  },
  "dependencies": {
    "dotenv": "^17.2.1"
  },
  "version": "1.0.0"
}
\end{lstlisting}

%==============================================================================
\subsection*{2.2 ZK回路(Circom)}
%==============================================================================

\begin{lstlisting}[language=C, caption=circuits/circuit.circom(軌跡検証回路)]
pragma circom 2.0.0;
include "circomlib/circuits/comparators.circom";

template TrajectoryCheck() {
    signal input zones[4];
    signal input timestamps[4];
    signal output result;

    var MAX_DURATION = 15;
    var EXPECTED_ZONES[4] = [1, 2, 3, 4];

    component zoneChecks[4];
    for (var i = 0; i < 4; i++) {
        zoneChecks[i] = IsEqual();
        zoneChecks[i].in[0] <== zones[i];
        zoneChecks[i].in[1] <== EXPECTED_ZONES[i];
    }

    component tsChecks[3];
    for (var i = 0; i < 3; i++) {
        tsChecks[i] = LessEqThan(64);
        tsChecks[i].in[0] <== timestamps[i];
        tsChecks[i].in[1] <== timestamps[i + 1];
    }

    signal totalTime;
    totalTime <== timestamps[3] - timestamps[0];

    component timeCheck = LessEqThan(64);
    timeCheck.in[0] <== totalTime;
    timeCheck.in[1] <== MAX_DURATION;

    signal all1;
    signal all2;
    signal all3;
    signal all4;
    signal all5;
    signal all6;
    signal all7;
    signal allPass;

    all1 <== zoneChecks[0].out * zoneChecks[1].out;
    all2 <== zoneChecks[2].out * zoneChecks[3].out;
    all3 <== tsChecks[0].out * tsChecks[1].out;
    all4 <== tsChecks[2].out * timeCheck.out;

    all5 <== all1 * all2;
    all6 <== all3 * all4;
    all7 <== all5 * all6;

    allPass <== all7;

    result <== allPass;
}

component main = TrajectoryCheck();
\end{lstlisting}

%==============================================================================
\subsection*{2.3 トラッキングシステム(Python)}
%==============================================================================

\begin{lstlisting}[language=Python, caption=tracking/zone\_definitions.py(ゾーン定義)]
# 640x480 を 横4列×縦3行 グリッドに分割
# 各セル: 160x160 ピクセル (640/4 = 160, 480/3 = 160)
ZONES = {
    # 1段目 (y=0-160)
    "zone_A": ((0,   0),   (160, 160)),
    "zone_B": ((160, 0),   (320, 160)),
    "zone_C": ((320, 0),   (480, 160)),
    "zone_D": ((480, 0),   (640, 160)),

    # 2段目 (y=160-320)
    "zone_E": ((0,   160), (160, 320)),
    "zone_F": ((160, 160), (320, 320)),
    "zone_G": ((320, 160), (480, 320)),
    "zone_H": ((480, 160), (640, 320)),

    # 3段目 (y=320-480)
    "zone_I": ((0,   320), (160, 480)),
    "zone_J": ((160, 320), (320, 480)),
    "zone_K": ((320, 320), (480, 480)),
    "zone_L": ((480, 320), (640, 480)),
}

# ゾーンごとの色設定 (B, G, R)
ZONE_COLORS = {
    "zone_A": (255, 255, 0),    # シアン
    "zone_B": (0, 255, 0),      # 緑
    "zone_C": (0, 0, 255),      # 赤
    "zone_D": (128, 128, 128),  # グレー
    "zone_E": (255, 0, 255),    # マゼンタ
    "zone_F": (0, 255, 255),    # 黄色
    "zone_G": (255, 128, 0),    # オレンジ
    "zone_H": (128, 0, 255),    # 紫
    "zone_I": (255, 0, 128),    # ピンク
    "zone_J": (0, 128, 255),    # オレンジ系
    "zone_K": (128, 255, 0),    # 黄緑
    "zone_L": (0, 128, 128),    # ダークシアン
}
\end{lstlisting}

\begin{lstlisting}[language=Python, caption=tracking/zone\_assigner.py(ゾーン判定)]
from .zone_definitions import ZONES

def get_zone(x, y):
    for name, ((x1, y1), (x2, y2)) in ZONES.items():
        if x1 <= x <= x2 and y1 <= y <= y2:
            return name
    return "unknown"
\end{lstlisting}

\begin{lstlisting}[language=Python, caption=tracking/trajectory\_manager.py(軌跡管理)]
import json
import os

# ゾーン履歴(ゾーン+タイムスタンプ)を記録
trajectories = {}

# 座標は描画用のみ
temp_coords = {}

def get_trajectory(track_id):
    return temp_coords.get(track_id, [])

def save_trajectory(extra_filepath: str | None = None):
    """
    常に data/trajectories.json(ZKP用の既存パス)に保存する。
    extra_filepath が指定されているときは、
    そのパスにも同じ内容を追加で保存する。
    """
    canonical_path = "data/trajectories.json"
    os.makedirs(os.path.dirname(canonical_path), exist_ok=True)
    with open(canonical_path, "w") as f:
        json.dump(trajectories, f, indent=2)

    if extra_filepath is not None:
        os.makedirs(os.path.dirname(extra_filepath), exist_ok=True)
        with open(extra_filepath, "w") as f:
            json.dump(trajectories, f, indent=2)

def update_trajectory(track_id, x, y, zone_name, timestamp):
    if track_id not in temp_coords:
        temp_coords[track_id] = []
    temp_coords[track_id].append((x, y))

    if track_id not in trajectories:
        trajectories[track_id] = []
        last_zone = None
    else:
        last_zone = trajectories[track_id][-1]["zone"] if trajectories[track_id] else None

    if zone_name != last_zone:
        trajectories[track_id].append({
            "zone": zone_name,
            "timestamp": timestamp
        })
\end{lstlisting}

%==============================================================================
\subsection*{2.4 物体検出・追跡(YOLO + DeepSORT)}
%==============================================================================

\begin{lstlisting}[language=Python, caption=detectors/yolo\_deepsort.py]
from ultralytics import YOLO
from deep_sort_realtime.deepsort_tracker import DeepSort

CONF_THRESHOLD = 0.8

def extract_person_boxes(results):
    boxes = []
    for box in results[0].boxes:
        if int(box.cls[0]) == 0:
            conf = float(box.conf[0])
            if conf < CONF_THRESHOLD:
                continue

            x1, y1, x2, y2 = map(int, box.xyxy[0])
            boxes.append(([x1, y1, x2 - x1, y2 - y1], conf, 'person'))
    return boxes

class YOLODeepSORTTracker:
    def __init__(self, yolo_model_path):
        self.model = YOLO(yolo_model_path)
        self.model.to("cuda")
        self.tracker = DeepSort()

    def update(self, frame):
        results = self.model(frame)
        person_boxes = extract_person_boxes(results)
        tracks = self.tracker.update_tracks(person_boxes, frame=frame)
        return tracks

    def active_ids(self):
        return [track.track_id for track in self.tracker.tracker.tracks if track.is_confirmed()]
\end{lstlisting}

%==============================================================================
\subsection*{2.5 メインプログラム(カメラトラッキング)}
%==============================================================================

\begin{lstlisting}[language=Python, caption=runtime/main.py]
import cv2
import time
import argparse
from datetime import datetime

from detectors.yolo_deepsort import YOLODeepSORTTracker
from tracking.trajectory_manager import update_trajectory, get_trajectory, save_trajectory


def build_grid_zones(width: int, height: int, cols: int, rows: int):
    """画面を cols x rows に均等分割したZONESを生成"""
    cell_w = width // cols
    cell_h = height // rows

    zones = {}
    for r in range(rows):
        for c in range(cols):
            x1 = c * cell_w
            y1 = r * cell_h
            x2 = (c + 1) * cell_w if c < cols - 1 else width
            y2 = (r + 1) * cell_h if r < rows - 1 else height
            zones[f"zone_{r}_{c}"] = ((x1, y1), (x2, y2))
    return zones


def get_zone(zones, x, y):
    """座標(x,y)が属するゾーン名を返す"""
    for name, ((x1, y1), (x2, y2)) in zones.items():
        if x1 <= x < x2 and y1 <= y < y2:
            return name
    return None


def draw_zones(frame, zones, thickness=1):
    """ゾーン枠と名前を描画"""
    grid_color = (170, 170, 170)
    for zone_name, ((x1, y1), (x2, y2)) in zones.items():
        cv2.rectangle(frame, (x1, y1), (x2, y2), grid_color, thickness)
        cv2.putText(frame, zone_name, (x1 + 5, y1 + 18),
                    cv2.FONT_HERSHEY_SIMPLEX, 0.5, grid_color, 1, cv2.LINE_AA)


def main():
    parser = argparse.ArgumentParser()
    parser.add_argument("--camera", type=int, default=0)
    parser.add_argument("--width", type=int, default=640)
    parser.add_argument("--height", type=int, default=480)
    parser.add_argument("--cols", type=int, default=3)
    parser.add_argument("--rows", type=int, default=4)
    parser.add_argument("--trial", type=str, default=None)
    parser.add_argument("--model", type=str, default="models/yolov8n.pt")
    parser.add_argument("--foot_mode", type=str, default="bottom",
                        choices=["bottom", "lower20"])
    parser.add_argument("--show_traj", action="store_true")
    args = parser.parse_args()

    trial_name = args.trial or datetime.now().strftime("%Y%m%d_%H%M%S")
    print(f"=== Trial: {trial_name} | Grid: {args.cols}x{args.rows} ===")

    cap = cv2.VideoCapture(args.camera)
    cap.set(cv2.CAP_PROP_FRAME_WIDTH, args.width)
    cap.set(cv2.CAP_PROP_FRAME_HEIGHT, args.height)

    if not cap.isOpened():
        print("カメラが開けませんでした")
        return

    ret, frame = cap.read()
    if not ret:
        print("初期フレームの取得に失敗しました")
        return

    H, W = frame.shape[:2]
    zones = build_grid_zones(W, H, args.cols, args.rows)

    tracker = YOLODeepSORTTracker(args.model)

    start_time = time.time()
    frame_count = 0
    fps_min = 9999.0

    try:
        while True:
            ret, frame = cap.read()
            if not ret:
                break

            draw_zones(frame, zones, thickness=1)
            tracks = tracker.update(frame)

            for track in tracks:
                if not track.is_confirmed():
                    continue

                track_id = track.track_id
                x1, y1, x2, y2 = map(int, track.to_ltrb())

                foot_x = int((x1 + x2) / 2)
                if args.foot_mode == "bottom":
                    foot_y = y2
                else:
                    foot_y = int(y2 - 0.2 * (y2 - y1))

                timestamp = datetime.now().isoformat()
                zone_name = get_zone(zones, foot_x, foot_y)

                update_trajectory(f"person_{track_id}", foot_x, foot_y, zone_name, timestamp)

                cv2.rectangle(frame, (x1, y1), (x2, y2), (255, 255, 0), 2)
                cv2.circle(frame, (foot_x, foot_y), 4, (255, 0, 0), -1)
                cv2.putText(frame, f"ID: {track_id}", (x1, y1 - 10),
                            cv2.FONT_HERSHEY_SIMPLEX, 0.6, (0, 255, 0), 2)

            if args.show_traj:
                for tid in tracker.active_ids():
                    traj = get_trajectory(f"person_{tid}")
                    for i in range(1, len(traj)):
                        cv2.line(frame, traj[i - 1], traj[i], (255, 0, 255), 2)

            frame_count += 1
            elapsed = time.time() - start_time
            fps = frame_count / max(elapsed, 1e-6)
            fps_min = min(fps_min, fps)
            cv2.putText(frame, f"FPS(avg): {fps:.2f}", (10, 30),
                        cv2.FONT_HERSHEY_SIMPLEX, 0.7, (0, 255, 255), 2)

            cv2.imshow("YOLO + DeepSORT Tracking", frame)

            key = cv2.waitKey(1) & 0xFF
            if key == ord("q"):
                break

    except KeyboardInterrupt:
        print("Ctrl+C で終了しました")

    finally:
        cap.release()
        save_trajectory(filename=f"data/trajectories_{trial_name}.json")
        cv2.destroyAllWindows()
        print(f"Saved: data/trajectories_{trial_name}.json")
        print(f"FPS avg: {fps:.2f} | FPS min: {fps_min:.2f}")


if __name__ == "__main__":
    main()
\end{lstlisting}

%==============================================================================
\subsection*{2.6 データ変換(軌跡→Circom入力)}
%==============================================================================

\begin{lstlisting}[language=Python, caption=utils/convert.py]
import json
import sys
from datetime import datetime

# ゾーン名を数値に変換
zone_map = {
    # 1段目
    "zone_A": 1,
    "zone_B": 2,
    "zone_C": 3,
    "zone_D": 4,
    # 2段目
    "zone_E": 5,
    "zone_F": 6,
    "zone_G": 7,
    "zone_H": 8,
    # 3段目
    "zone_I": 9,
    "zone_J": 10,
    "zone_K": 11,
    "zone_L": 12,
}

# ISO形式 -> UNIX秒に変換
def iso_to_unix(timestamp_str):
    return int(datetime.fromisoformat(timestamp_str).timestamp())

# 最新の zone_D 到達者の軌跡を抽出して input.json に変換
def extract_latest_zone_d(file_path, output_path):
    with open(file_path, 'r') as f:
        all_data = json.load(f)

    latest_person = None
    latest_d_time = None

    for person_id, records in all_data.items():
        for idx, entry in enumerate(records):
            if entry["zone"] == "zone_D":
                ts = datetime.fromisoformat(entry["timestamp"])
                if latest_d_time is None or ts > latest_d_time:
                    latest_d_time = ts
                    latest_person = (person_id, records[:idx + 1])

    if latest_person is None:
        print("[!] zone_D に到達した人物がいませんでした。")
        return

    person_id, trajectory = latest_person
    zones = [zone_map[pt["zone"]] for pt in trajectory]
    timestamps = [iso_to_unix(pt["timestamp"]) for pt in trajectory]

    output = {
        "zones": zones,
        "timestamps": timestamps
    }

    with open(output_path, 'w') as f:
        json.dump(output, f, indent=2)

    print(f"[OK] {person_id} のデータを Circom 用に変換しました -> {output_path}")

if __name__ == "__main__":
    if len(sys.argv) != 3:
        print("Usage: convert.py <input_json> <output_json>")
        sys.exit(1)

    input_json = sys.argv[1]
    output_json = sys.argv[2]

    extract_latest_zone_d(input_json, output_json)
\end{lstlisting}

%==============================================================================
\subsection*{2.7 Witness生成(JavaScript)}
%==============================================================================

\begin{lstlisting}[language=JavaScript, caption=circuits/circuit\_js/generate\_witness.js]
const wc  = require("./witness_calculator.js");
const { readFileSync, writeFile } = require("fs");

if (process.argv.length != 5) {
    console.log("Usage: node generate_witness.js <file.wasm> <input.json> <output.wtns>");
} else {
    const input = JSON.parse(readFileSync(process.argv[3], "utf8"));

    const buffer = readFileSync(process.argv[2]);
    wc(buffer).then(async witnessCalculator => {
        const buff = await witnessCalculator.calculateWTNSBin(input, 0);
        writeFile(process.argv[4], buff, function(err) {
            if (err) throw err;
        });
    });
}
\end{lstlisting}

%==============================================================================
\subsection*{2.8 ZKP生成パイプライン(シェルスクリプト)}
%==============================================================================

\begin{lstlisting}[language=bash, caption=generate\_proof.sh]
#!/bin/bash
# ==============================================================
#  ZK Pipeline (preprocess + witness + prove + verify)
# ==============================================================

set -euo pipefail
RPI_NOTIFY_URL="http://192.168.101.9:5000/zkp-stored"

# --- 実行コマンド ---
PYTHON_EXEC="${PYTHON_EXEC:-python3}"
NODE_EXEC="${NODE_EXEC:-$(command -v node || echo /usr/local/bin/node)}"
SNARKJS_EXEC="${SNARKJS_EXEC:-$(command -v snarkjs || echo /usr/local/bin/snarkjs)}"

# --- プロジェクトルート ---
BASE_DIR="$(cd "$(dirname "$0")" && pwd)"

# --- ディレクトリ・ファイル ---
CIRCUITS_DIR="$BASE_DIR/circuits"
DATA_DIR="$BASE_DIR/data"
UTILS_DIR="$BASE_DIR/utils"

CIRCUIT_NAME="${CIRCUIT_NAME:-circuit}"

# 入出力
TRAJECTORIES_JSON="${TRAJECTORIES_JSON:-$DATA_DIR/trajectories.json}"
INPUT_JSON="${INPUT_JSON:-$CIRCUITS_DIR/input.json}"
WITNESS_WTNS="$BASE_DIR/output/witness.wtns"
PROOF_JSON="$BASE_DIR/output/proof.json"
PUBLIC_JSON="$BASE_DIR/output/public.json"

# circom 生成物
WASM_PATH="${WASM_PATH:-$CIRCUITS_DIR/${CIRCUIT_NAME}_js/${CIRCUIT_NAME}.wasm}"
GEN_WITNESS_JS="${GEN_WITNESS_JS:-$CIRCUITS_DIR/${CIRCUIT_NAME}_js/generate_witness.js}"
ZKEY_PATH="${ZKEY_PATH:-$CIRCUITS_DIR/${CIRCUIT_NAME}_final.zkey}"
VK_PATH="${VK_PATH:-$CIRCUITS_DIR/verification_key.json}"

CONVERT_PY="${CONVERT_PY:-$UTILS_DIR/convert.py}"

OUTPUT_DIR="$BASE_DIR/output"
mkdir -p "$OUTPUT_DIR"

echo "=============================================================="
echo " Zero-Knowledge Proof Pipeline"
echo "=============================================================="

# --- 依存コマンドチェック ---
command -v "$PYTHON_EXEC" >/dev/null 2>&1 || { echo "error: $PYTHON_EXEC が見つかりません"; exit 1; }
[ -x "$NODE_EXEC" ] || { echo "error: node が見つかりません"; exit 1; }
[ -x "$SNARKJS_EXEC" ] || { echo "error: snarkjs が見つかりません"; exit 1; }

# --- 必須ファイルチェック ---
[ -f "$TRAJECTORIES_JSON" ] || { echo "error: $TRAJECTORIES_JSON がありません"; exit 1; }
[ -f "$CONVERT_PY" ] || { echo "error: $CONVERT_PY がありません"; exit 1; }
[ -f "$WASM_PATH" ] || { echo "error: $WASM_PATH がありません"; exit 1; }
[ -f "$GEN_WITNESS_JS" ] || { echo "error: $GEN_WITNESS_JS がありません"; exit 1; }
[ -f "$ZKEY_PATH" ] || { echo "error: $ZKEY_PATH がありません"; exit 1; }
[ -f "$VK_PATH" ] || { echo "error: $VK_PATH がありません"; exit 1; }

# [前処理] trajectories.json -> input.json
echo "[前処理] $TRAJECTORIES_JSON -> $INPUT_JSON を生成..."
"$PYTHON_EXEC" "$CONVERT_PY" "$TRAJECTORIES_JSON" "$INPUT_JSON"

# [1] Witness の計算
echo "[ステップ 1/3] Witness を計算中..."
"$NODE_EXEC" "$GEN_WITNESS_JS" "$WASM_PATH" "$INPUT_JSON" "$WITNESS_WTNS"

# [2] Proof の生成
echo "[ステップ 2/3] Proof を生成中..."
"$SNARKJS_EXEC" groth16 prove "$ZKEY_PATH" "$WITNESS_WTNS" "$PROOF_JSON" "$PUBLIC_JSON"

# [3] 検証
echo "[ステップ 3/3] 生成した Proof を検証中..."
VERIFY_OUT="$("$SNARKJS_EXEC" groth16 verify "$VK_PATH" "$PUBLIC_JSON" "$PROOF_JSON" || true)"
if [[ "$VERIFY_OUT" == *"OK"* ]]; then
  echo "検証に成功しました!"
else
  echo "検証に失敗しました。"
  echo "$VERIFY_OUT"
  exit 1
fi

echo "全てのプロセスが正常に完了しました。"

# =======================
# ブロックチェーン連携
# =======================
RUN_CHAIN="${RUN_CHAIN:-1}"
BLOCKCHAIN_DIR="${BLOCKCHAIN_DIR:-$BASE_DIR/../blockchain_root}"

if [[ "$RUN_CHAIN" == "1" ]]; then
  echo "ブロックチェーン連携を実行します..."

  [ -d "$BLOCKCHAIN_DIR" ] || { echo "error: blockchain_root が見つかりません"; exit 1; }

  mkdir -p "$BLOCKCHAIN_DIR/proof"
  cp "$PROOF_JSON"  "$BLOCKCHAIN_DIR/proof/proof.json"
  cp "$PUBLIC_JSON" "$BLOCKCHAIN_DIR/proof/public.json"

  if [ -x "$BLOCKCHAIN_DIR/StoreAndVerify.sh" ]; then
    pushd "$BLOCKCHAIN_DIR" >/dev/null
    ./StoreAndVerify.sh
    popd >/dev/null
    echo "ブロックチェーン側の検証+保存まで完了しました"
  fi
fi

# --- HTTP(Webhook) 通知 ---
EVENT_JSON='{"event": "zkp_stored", "timestamp": "'"$(date -Is)"'"}'

if [[ -n "${RPI_NOTIFY_URL:-}" ]]; then
  echo "ラズパイへ HTTP 通知: $RPI_NOTIFY_URL"
  curl -sS -X POST "$RPI_NOTIFY_URL" \
    -H 'Content-Type: application/json' \
    -d "$EVENT_JSON" || echo "HTTP通知に失敗しました"
fi
\end{lstlisting}

%==============================================================================
\subsection*{2.9 スマートコントラクト(Solidity)}
%==============================================================================

\begin{lstlisting}[language=Solidity, caption=contracts/Verifier.sol(Groth16検証コントラクト)]
// SPDX-License-Identifier: GPL-3.0
pragma solidity >=0.7.0 <0.9.0;

contract Groth16Verifier {
    // Scalar field size
    uint256 constant r = 21888242871839275222246405745257275088548364400416034343698204186575808495617;
    // Base field size
    uint256 constant q = 21888242871839275222246405745257275088696311157297823662689037894645226208583;

    // Verification Key data
    uint256 constant alphax = 20491192805390485299153009773594534940189261866228447918068658471970481763042;
    uint256 constant alphay = 9383485363053290200918347156157836566562967994039712273449902621266178545958;
    uint256 constant betax1 = 4252822878758300859123897981450591353533073413197771768651442665752259397132;
    uint256 constant betax2 = 6375614351688725206403948262868962793625744043794305715222011528459656738731;
    uint256 constant betay1 = 21847035105528745403288232691147584728191162732299865338377159692350059136679;
    uint256 constant betay2 = 10505242626370262277552901082094356697409835680220590971873171140371331206856;
    uint256 constant gammax1 = 11559732032986387107991004021392285783925812861821192530917403151452391805634;
    uint256 constant gammax2 = 10857046999023057135944570762232829481370756359578518086990519993285655852781;
    uint256 constant gammay1 = 4082367875863433681332203403145435568316851327593401208105741076214120093531;
    uint256 constant gammay2 = 8495653923123431417604973247489272438418190587263600148770280649306958101930;
    uint256 constant deltax1 = 2263397671767056286221887566256319682169787836886194803533788119416677936226;
    uint256 constant deltax2 = 4246651844915815373630795255234623463537658050783473939235453788256760825446;
    uint256 constant deltay1 = 494182514580644992626377724493218725935212809218998442440704025136732564752;
    uint256 constant deltay2 = 5651219803589856862797856707670733934183624624433905687405214668122259936915;

    uint256 constant IC0x = 2727670057803473048354643319762904847867319821993807729518486185744946288351;
    uint256 constant IC0y = 8930886798563919431690935525434869581893047636074101377527165946415727264319;
    uint256 constant IC1x = 9420041431406619851470766177137173535219815147825583567234218496145543717785;
    uint256 constant IC1y = 16543940149804679353570736123148571985482696626789387222901483636640998498683;

    uint16 constant pVk = 0;
    uint16 constant pPairing = 128;
    uint16 constant pLastMem = 896;

    function verifyProof(
        uint[2] calldata _pA,
        uint[2][2] calldata _pB,
        uint[2] calldata _pC,
        uint[1] calldata _pubSignals
    ) public view returns (bool) {
        assembly {
            function checkField(v) {
                if iszero(lt(v, r)) {
                    mstore(0, 0)
                    return(0, 0x20)
                }
            }

            function g1_mulAccC(pR, x, y, s) {
                let success
                let mIn := mload(0x40)
                mstore(mIn, x)
                mstore(add(mIn, 32), y)
                mstore(add(mIn, 64), s)

                success := staticcall(sub(gas(), 2000), 7, mIn, 96, mIn, 64)
                if iszero(success) {
                    mstore(0, 0)
                    return(0, 0x20)
                }

                mstore(add(mIn, 64), mload(pR))
                mstore(add(mIn, 96), mload(add(pR, 32)))

                success := staticcall(sub(gas(), 2000), 6, mIn, 128, pR, 64)
                if iszero(success) {
                    mstore(0, 0)
                    return(0, 0x20)
                }
            }

            function checkPairing(pA, pB, pC, pubSignals, pMem) -> isOk {
                let _pPairing := add(pMem, pPairing)
                let _pVk := add(pMem, pVk)

                mstore(_pVk, IC0x)
                mstore(add(_pVk, 32), IC0y)

                g1_mulAccC(_pVk, IC1x, IC1y, calldataload(add(pubSignals, 0)))

                mstore(_pPairing, calldataload(pA))
                mstore(add(_pPairing, 32), mod(sub(q, calldataload(add(pA, 32))), q))

                mstore(add(_pPairing, 64), calldataload(pB))
                mstore(add(_pPairing, 96), calldataload(add(pB, 32)))
                mstore(add(_pPairing, 128), calldataload(add(pB, 64)))
                mstore(add(_pPairing, 160), calldataload(add(pB, 96)))

                mstore(add(_pPairing, 192), alphax)
                mstore(add(_pPairing, 224), alphay)

                mstore(add(_pPairing, 256), betax1)
                mstore(add(_pPairing, 288), betax2)
                mstore(add(_pPairing, 320), betay1)
                mstore(add(_pPairing, 352), betay2)

                mstore(add(_pPairing, 384), mload(add(pMem, pVk)))
                mstore(add(_pPairing, 416), mload(add(pMem, add(pVk, 32))))

                mstore(add(_pPairing, 448), gammax1)
                mstore(add(_pPairing, 480), gammax2)
                mstore(add(_pPairing, 512), gammay1)
                mstore(add(_pPairing, 544), gammay2)

                mstore(add(_pPairing, 576), calldataload(pC))
                mstore(add(_pPairing, 608), calldataload(add(pC, 32)))

                mstore(add(_pPairing, 640), deltax1)
                mstore(add(_pPairing, 672), deltax2)
                mstore(add(_pPairing, 704), deltay1)
                mstore(add(_pPairing, 736), deltay2)

                let success := staticcall(sub(gas(), 2000), 8, _pPairing, 768, _pPairing, 0x20)
                isOk := and(success, mload(_pPairing))
            }

            let pMem := mload(0x40)
            mstore(0x40, add(pMem, pLastMem))

            checkField(calldataload(add(_pubSignals, 0)))

            let isValid := checkPairing(_pA, _pB, _pC, _pubSignals, pMem)

            mstore(0, isValid)
            return(0, 0x20)
        }
    }
}
\end{lstlisting}

\begin{lstlisting}[language=Solidity, caption=contracts/ATMVerifier.sol(ZKP保存コントラクト)]
// SPDX-License-Identifier: MIT
pragma solidity ^0.8.0;

import "./Verifier.sol";

contract ZKPStorage {
    Groth16Verifier public verifier;

    constructor(address _verifier) {
        verifier = Groth16Verifier(_verifier);
    }

    struct Proof {
        uint[2] a;
        uint[2][2] b;
        uint[2] c;
        uint[1] input;
        bool verified;
    }

    mapping(address => Proof) private proofs;

    event ProofStored(address indexed user, bool verified);

    function verifyAndStoreProof(
        uint[2] memory a,
        uint[2][2] memory b,
        uint[2] memory c,
        uint inputValue
    ) public returns (bool) {
        uint[1] memory input = [inputValue];

        bool isValid = verifier.verifyProof(a, b, c, input);

        proofs[msg.sender] = Proof(a, b, c, input, isValid);

        emit ProofStored(msg.sender, isValid);
        return isValid;
    }

    function getMyProof() public view returns (
        uint[2] memory a,
        uint[2][2] memory b,
        uint[2] memory c,
        uint[1] memory input,
        bool verified
    ) {
        Proof memory p = proofs[msg.sender];
        return (p.a, p.b, p.c, p.input, p.verified);
    }
}
\end{lstlisting}

%==============================================================================
\subsection*{2.10 ブロックチェーン操作スクリプト}
%==============================================================================

\begin{lstlisting}[language=JavaScript, caption=scripts/deploy.js(コントラクトデプロイ)]
// scripts/deploy.js
async function main() {
  const Verifier = await ethers.getContractFactory("Groth16Verifier");
  const verifier = await Verifier.deploy();
  await verifier.waitForDeployment();

  console.log("Verifier deployed to:", await verifier.getAddress());

  const ZKPStorage = await ethers.getContractFactory("ZKPStorage");
  const zkpStorage = await ZKPStorage.deploy(await verifier.getAddress());
  await zkpStorage.waitForDeployment();

  console.log("ZKPStorage deployed to:", await zkpStorage.getAddress());
}

main().catch((error) => {
  console.error(error);
  process.exitCode = 1;
});
\end{lstlisting}

\begin{lstlisting}[language=JavaScript, caption=scripts/storeAndVerify.js(証明の検証・保存)]
const hre = require("hardhat");
const fs = require("fs");

async function main() {
  const zkpStorageAddress = "0xb4f4e63DF760209E0a4B470088B5268C2843B7e0";

  const [signer] = await hre.ethers.getSigners();
  console.log("Using signer:", await signer.getAddress());

  const ZKPStorage = await hre.ethers.getContractFactory("ZKPStorage", signer);
  const zkpStorage = await ZKPStorage.attach(zkpStorageAddress);

  const proof = JSON.parse(fs.readFileSync("./proof/proof.json"));
  const pub = JSON.parse(fs.readFileSync("./proof/public.json"));

  const a = [
    BigInt(proof.pi_a[0]),
    BigInt(proof.pi_a[1])
  ];

  const b = [
    [BigInt(proof.pi_b[0][1]), BigInt(proof.pi_b[0][0])],
    [BigInt(proof.pi_b[1][1]), BigInt(proof.pi_b[1][0])]
  ];

  const c = [
    BigInt(proof.pi_c[0]),
    BigInt(proof.pi_c[1])
  ];

  const inputValue = BigInt(pub[0]);

  console.log("Verifying and storing proof...");

  const start = Date.now();

  const tx = await zkpStorage.verifyAndStoreProof(a, b, c, inputValue);
  const receipt = await tx.wait();

  const end = Date.now();
  const duration = end - start;

  console.log(`Verification and storage took ${duration} ms`);
  console.log("TX Hash:", receipt.transactionHash);

  const storedProof = await zkpStorage.getMyProof();
  console.log("Stored input:", storedProof.input.map(v => v.toString()));
  console.log("Verified:", storedProof.verified);
}

main().catch((error) => {
  console.error(error);
  process.exitCode = 1;
});
\end{lstlisting}

\begin{lstlisting}[language=bash, caption=StoreAndVerify.sh(ブロックチェーン保存ラッパー)]
#!/bin/bash
set -euo pipefail

SCRIPT_DIR="$(cd "$(dirname "$0")" && pwd)"
BLOCKCHAIN_DIR="$SCRIPT_DIR"
PROJECT_ROOT="${PROJECT_ROOT:-$SCRIPT_DIR/../project_root}"

echo "=============================================================="
echo " Blockchain Store & Verify"
echo "=============================================================="

PROOF_FILE="$BLOCKCHAIN_DIR/proof/proof.json"
PUBLIC_FILE="$BLOCKCHAIN_DIR/proof/public.json"

if [ ! -f "$PROOF_FILE" ] || [ ! -f "$PUBLIC_FILE" ]; then
  echo "エラー: proof.json または public.json が見つかりません"
  exit 1
fi

echo "ブロックチェーン上で検証+保存中..."
cd "$BLOCKCHAIN_DIR"
npx hardhat run scripts/storeAndVerify.js --network sepolia

echo "ブロックチェーンへの保存が完了しました"
\end{lstlisting}

%==============================================================================
\subsection*{2.11 Hardhat設定}
%==============================================================================

\begin{lstlisting}[language=JavaScript, caption=hardhat.config.js]
require("dotenv").config();
require("@nomicfoundation/hardhat-toolbox");

/** @type import('hardhat/config').HardhatUserConfig */
module.exports = {
  solidity: "0.8.28",
  networks: {
    sepolia: {
      url: process.env.SEPOLIA_RPC_URL,
      accounts: [process.env.PRIVATE_KEY]
    }
  }
};
\end{lstlisting}

%==============================================================================
\subsection*{2.12 Flaskサーバ(ATMからの通知受信)}
%==============================================================================

\begin{lstlisting}[language=Python, caption=flask\_server/server.py]
from flask import Flask, request, jsonify
import subprocess
import os

app = Flask(__name__)

@app.route('/alert', methods=['POST'])
def handle_alert():
    data = request.json
    print("ATMからの通知を受信:", data)

    try:
        current_script_path = os.path.dirname(os.path.abspath(__file__))
        project_root = os.path.abspath(os.path.join(current_script_path, '..'))
        shell_script_path = os.path.join(project_root, 'generate_proof.sh')

        print(f"実行するスクリプトの絶対パス: {shell_script_path}")
        result = subprocess.run(
            ["bash", shell_script_path],
            check=True,
            capture_output=True,
            text=True,
            timeout=300
        )
        print("ZKP生成成功:\n", result.stdout)
        return jsonify({"status": "zkp_success"}), 200

    except subprocess.CalledProcessError as e:
        print("ZKP生成失敗:")
        print("   STDOUT:", e.stdout)
        print("   STDERR:", e.stderr)
        return jsonify({"status": "zkp_failed", "error": e.stderr}), 500
    except FileNotFoundError:
        print(f"シェルスクリプトが見つかりません: {shell_script_path}")
        return jsonify({"status": "script_not_found"}), 500
    except Exception as e:
        print(f"予期せぬエラーが発生しました: {e}")
        return jsonify({"status": "unknown_error", "message": str(e)}), 500

if __name__ == '__main__':
    app.run(host='0.0.0.0', port=5000)
\end{lstlisting}
